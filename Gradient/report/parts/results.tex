\documentclass[../body.tex]{subfiles}
\begin{document}
В качестве начального приближения была взята точка $(0,0)$ для всех опытов.
\begin{table}[H]
    \centering
    \begin{tabular}{|c|c|c|c|}
        \hline
        Точность & Точка минимума & Итерации\\\hline
        $10^{-1}$ & (-0.42474394, -0.31855834) & 4\\\hline
        $10^{-2}$ & (-0.4284002, -0.32130114) & 6\\\hline
        $10^{-3}$ & (-0.42879686, -0.32159787) & 8\\\hline
        $10^{-4}$ & (-0.42883981, -0.3216299) & 10\\\hline
    \end{tabular}
    \caption{Результат градиентного метода наискорейшего спуска}
\end{table}

\begin{table}[H]
    \centering
    \begin{tabular}{|c|c|c|c|}
        \hline
        Точность & Точка минимума & Итерации\\\hline
        $10^{-1}$ & (-0.42867724, -0.32150793) & 3\\\hline
        $10^{-2}$ & (-0.42884497, -0.32163373) & 4\\\hline
        $10^{-3}$ & (-0.42884497, -0.32163373) & 4\\\hline
        $10^{-4}$ & (-0.42884501, -0.32163376) & 5\\\hline
    \end{tabular}
    \caption{Результат градиентного метода Ньютона}
\end{table}

\begin{table}[H]
    \centering
    \begin{tabular}{|c|c|c|c|}
        \hline
        Точность & Точка минимума & Итерации\\\hline
        $10^{-1}$ & (-0.42869936, -0.32144532) & 4\\\hline
        $10^{-2}$ & (-0.42886025, -0.32158642) & 5\\\hline
        $10^{-3}$ & (-0.42886025, -0.32158642) & 5\\\hline
        $10^{-4}$ & (-0.42884501, -0.32163375) & 7\\\hline
    \end{tabular}
    \caption{Результат градиентного ДФП-метода}
\end{table}
\end{document}