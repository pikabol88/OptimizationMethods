\documentclass[main.tex]{subfiles}
\begin{document}
\subsection{Постановка задачи квадратичного программирования}
\textit{Задача квадратичного программирования} - задача минимизации квадратичной функции на выпуклом многогранном множестве $\Omega$ [3, 5]:
\begin{equation}
    F(x)=\frac{1}{2}<Dx,x>+<c,x>\rightarrow\min_{x\in\Omega},
    \label{task}
\end{equation}
где $N=\{1,...,n\}$, $c\in\mathbb{R}^n$, матрица $D[N,N]$ симметрична и положительно определена, а
\begin{equation}
    \Omega=
    \left\{
    \begin{matrix}
        & | & A[M_1,N]*x[N]\geq b[M_1]\\
        x[N] & | & A[M_2,N]*x[N]=b[M_2]\\
        & | & x[N_1]\geq0,N_1\subset N\\\
    \end{matrix}
    \right\}.
\end{equation}

\textit{Оптимальный план задачи} - вектор  $x^*\in\Omega:F(x^*)\leq F(x)\forall x\in\Omega$.

\textit{Двойственная задача квадратичного программирования:}
\begin{equation}
        G(y)=\frac{1}{2}<Qy,y>+<d,y>-\frac{1}{2}<D^{-1}c,c>\rightarrow\max_{y\in\Omega},
\end{equation}
гдe $c\geq0,Q=-AD^{-1}A^T,d=-b-AD^{-1}c$ [5, 13].

\textbf{Теорема 1:} Задача \eqref{task} разрешима $\leftrightarrow\Omega\not=\emptyset$ и целевая функция $F(x)$ ограничена снизу на множестве планов [3, 13].

\textbf{Теорема 2:} Пусть $\Omega\not=\emptyset$. Задача \eqref{task} разрешима, если выполнено хотя бы одно из условий [3, 13]:
\begin{enumerate}
    \item множество $\Omega$ ограничено
    \item матрица $D$ положительно определена
    \item линейная часть целевой функции $<c,x>$ ограничена снизу на $\Omega$
\end{enumerate}

\textbf{Лемма:} вектор $x^*\in\Omega$ - решение задачи \eqref{task} $\leftrightarrow<F'(x^*),(x-x^*)>\geq0\forall x\in\Omega$ [3, 16-17].

\textbf{Теорема Куна-Такера:} вектор $x^*[N]$ - решение задачи \eqref{task} $\leftrightarrow \exists u^*[M],y^*[N],v^*[N]$, удовлетворяющих условиям:
\begin{equation}
    \left\{
    \begin{array}{ll}
        D[N,N]*x^*[N]-A^T[N,M]*u^*[M]-y^*[N]=-c[N]\\
        A[M,N]*x^*[N]-v^*[M]=b[M]\\
        y^*[N_2]=0\\
        v^*[M_2]=0\\
        x^*[N_1]\geq0\\
        u^*[M_1]\geq0\\
        y^*[N_1]\geq0\\
        v^*[M_1]\geq0\\
        x^*[N_1]*y^*[N_1]=0\\
        u^*[M_1]*v^*[M_1]=0\\
    \end{array}
    \right.
\end{equation}
При этом $F(x^*)=\frac{1}{2}(<c,x^*>+<b,u^*>)$.

Таким образом, решение задачи \eqref{task} равносильно разрешению системы теоремы Куна-Такера. Если данная система несовместна, то исходная задача неразрешима, и наоборот [3, 17-21].
\end{document}
