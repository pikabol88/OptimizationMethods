\documentclass[main.tex]{subfiles}
\begin{document}
\begin{titlepage}
\begin{center}
	\begin{large}
		Санкт-Петербургский политехнический университет\\ Петра Великого\\		
		\vspace{\baselineskip}
		Институт прикладной математики и механики\\
		\textbf{Кафедра <<Прикладная математика>>}\\
	\end{large}
	\vfill
	\Large{{КУРСОВАЯ РАБОТА\\
	ПО ДИСЦИПЛИНЕ <<МЕТОДЫ ОПТИМИЗАЦИИ>>\\ <<СРАВНИТЕЛЬНЫЙ АНАЛИЗ МЕТОДОВ РЕШЕНИЯ ЗАДАЧИ КВАДРАТИЧНОГО ПРОГРАММИРОВАНИЯ ПРИ ИСПОЛЬЗОВАНИИ SVM В ЗАДАЧАХ РАСПОЗНОВАНИЯ>>}}
\end{center}
\vfill
\flushleft
Выполнили\\
студенты группы 3630102/80201
\flushright
Деркаченко А. О.\\
\textit{Классический метод машины опорных векторов, Методы формирования вектора признаков изображения лица для определении атрибутов личности, Сравнительный анализ}\\
Хрипунков Д. В.\\
\textit{Введение, Постановка задачи}\\
Войнова А. Н.\\
\textit{Модификации метода машины опорных векторов, Сравнительный анализ}\\
\flushleft
Руководитель\\
к. ф.-м. н., доц.\\
\flushright
Родионова Елена Александровна
\vfill
\centering{Санкт-Петербург \\ 2021}
\end{titlepage}
\end{document}