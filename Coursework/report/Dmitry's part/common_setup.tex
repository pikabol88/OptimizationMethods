\documentclass[12pt,a4paper]{article}

\usepackage[T1,T2A]{fontenc}
\usepackage[utf8]{inputenc}
\usepackage[english, russian]{babel}
\usepackage{indentfirst}
\usepackage{hyperref}
\usepackage{misccorr}
\usepackage{graphicx}
\usepackage{amsmath}
\usepackage{graphicx}
\usepackage{float}
\usepackage[left=20mm,right=10mm, top=20mm,bottom=20mm,bindingoffset=0mm]{geometry}

\setlength{\parskip}{6pt}
\DeclareGraphicsExtensions{.png}

\begin{document}
\section{Постановка задачи}
\subsection{Общая постановка задачи распознавания}
Рассмотрим общую постановку задачи на примере распознавания символов. Предположим, система распознавания получила на вход некоторый символ (паттерн) X, который нужно распознать.

Система может считывать скорость изменения закрашенной поверхности как функцию от времени X(t), называемую представлением символа X. Альтернативным вариантом можно считывать сигнал в дискретные моменты времени, в результате чего получается вектор \={X}. Также возможны переходы из представления в виде функции в векторное.

Предположим, что есть некоторое множество неперескающихся классов $\Omega=\{\bar{\omega_1}, ..., \bar{\omega_m}\}$, где каждое $\omega_i$ отвечает некоторому символу. Системе распознавания нужно отнести входящий символ X к какому-то из классов $\omega_i$. Для этого предпринимаются следующие шаги:
\begin{enumerate}
    \item Символ X считывается в представление X(t)
    \item Представление X(t) преобразуется в векторную форму \=X
    \item Из вектора \=X извлекаются информативные признаки и образуется вектор \^X
    \item Классификатор определяет, к какому классу относятся признаки \^X
\end{enumerate}

Для финального соотнесения к некоторому классу используется классификатор - набор правил для определения класса. После классификации, символ может быть определён к одному из существующих классов, или как неотносящийся ни к одному из них. Классификация может выполняться, например, посредством вычисления расстояния между классом и вектором признаков \^X.

Также в схеме распознавания может присутствовать блок обучения. Он выбирает учебные образы, которые заведомо распределены по классам. С их помощью можно сформировать правила классификации или определить наиболее информативные признаки.

Декомпозируя задачу распознавания на подзадачи, получается следующий набор:
\begin{enumerate}
    \item Математически описать образ \\
        Такое описание удобнее всего проводить в векторной форме. Образу X сопоставляется некоторый вектор $\bar{X}=(x_1, x_2, ...)^T$, где каждое $x_i$ - некоторый признак, а $\bar{X}$ - элемент конечномерного метрического векторного пространства. $\mathbb{X}$
    \item Выбрать информативные признаки \\
        Не все признаки символа могут быть одинаково полезны при распознавании. Задача состоит в том, чтобы определить минимально необходимый набор признаков, достаточных для распознавания символа. Этот набор система должна определить сама.
    \item Описать классы \\
        Необходимо задать границы классов. Это может быть проделано на этапе разработки или самой системой в ходе её работы.
    \item Определить методы классификации \\
        Нужно определить методы, по которым образы будут соотноситься некоторым классам.
    \item Определить оценку достоверности распознавания \\
        Оценка нужна для того, чтобы иметь возможность определить величину потерь при неправильной классификации
\end{enumerate}

Математически, задачу распознавания можно поставить так: \\
Дано множество образов $U$, отдельный образ обозначим $x \in U$. Из (возможно несчётного) множества признаков образов $x$ нужно выбрать конечное подмножество - пространство признаков. Пространство признаков конечномерное, линейное или метрическое, обозначим как $X$. Каждому образу соответствует элемент $\bar{x} \in U$ и оператор $P: U \longrightarrow X$ отображения образа в пространство признаков. Введём конечное множество классов $\Omega=\{\bar{\omega_1}, ..., \bar{\omega_m}\}$, дял которого верно, что $\cup_{i=1}^m\omega_i = U$ и $\omega_i \cap \omega_j = \varnothing, \forall i \not= j$. Для классификации образа $x \in U$ по классам из $\Omega$ нужно найти индикаторную функцию $g: U \longrightarrow Y$, $Y=\{y1, ..., y_m\}$, где $Y$ - множество меток класса. То есть $g(x)=y_i$, если $x \in \omega_i$. Поскольку в реальности мы работаем не с самим образами, а их признаками, то нужно найти решающую функцию $\tilde{g}: X \longrightarrow Y$ для $\bar{x}=P*x \in X$, то есть $\tilde{g}(\bar{x})=y_i$, если $\bar{x}=P*x \in \omega_i$. Поскольку множество $P^{-1}\bar{x}$, $\bar{x} \in X$ может иметь непустые пересечения с разными классами $\omega_i$, то функция $\tilde{g}(x)$ будет неоднозначной, тогда из неё нужно выделить однозначную ветвь, удовлетворяющую условиям оптимальности, например, минимальность ошибки неправильной классификации. На этапе обучения система известны некоторые пары $(\bar{x}_j, y_j)$, называемые прецендентами и множество $H = \{\bar{x_1}, ...,\bar{x_N}\}$, называемое обучающей выборкой. По множеству прецендентов $(H, Y)$ нужно построить решающую функцию $\tilde{g}(x)$, которая будет осуществлять классификацию.

\end{document}