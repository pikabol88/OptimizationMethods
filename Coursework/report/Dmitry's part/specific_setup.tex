\documentclass[12pt,a4paper]{article}
\begin{document}
\subsection{Постановка задачи по определению атрибутов личности по изображению лица}
Задача определения атрибутов личности по изображению лица - частный случай задачи обучения по прецендентам, описанной выше, где множество $U$ - множество нормализованных изображений [1, 28].

Для каждого из выбранных нами атрибутов будет выбрана соответствующая задача классификации в зависимости от множества меток классов $Y$ [1, 31]:
\begin{enumerate}
    \item Для атрибута "пол" будет бинарная классификация $Y = \{-1 , 1\}$
    \item Для атрибута "раса" множественная классификация $Y = \{-1, 0, 1\}$ для рас ``европеоидная``, ``монголоидная`` и ``негроидная``
    \item Для атрибута "возраст" восстановление регрессии $Y = [5, 100]$
\end{enumerate}

Эти задачи связаны друг с другом, так, например, задача множественной классификации может быть разложена на несколько задача бинарной классификации, а задача восстановления регрессии формулируется исходя из решения задачи множественной классификации [1, 31].

Задача определения атрибутов личности по изображению лица сводится к тому, что нужно определить метод формирования вектора признаков $P: U \longrightarrow X$ и решающую функцию $\tilde{g}: X \longrightarrow Y$ так, чтобы оценка достоверности распознавания была максимальной. Решение такой задачи получается сравнением решений из конечного множества частных оптимизационных задач, в которых выбраны разные способы формирования вектора признаков и решающей функции [1, 32].

\end{document}