\documentclass[../body.tex]{subfiles}
\begin{document}		
	\textbf{Симплекс метод} был предложен американским математиком Р.Данцигом в 1947 году, с тех пор не утратил свою актуальность, для нужд промышленности этим методом нередко решаются задачи линейного программирования с тысячами переменных и ограничений. \\
	\vspace{\baselineskip}
	
	Основные преимущества метода:
	\begin{itemize}
		\item Симплекс-метод является универсальным методом, которым можно решить любую задачу линейного программирования, в то время, как графический метод пригоден лишь для системы ограничений с двумя переменными.
		\item Решение будет гарантировано найдено за $O(2^n)$ операций, где n - это количество переменных.
		\item Не так хорош для больших задач, но есть множетсво улучшений базового симплекс-метода, которые компенсируют эту проблему.
	\end{itemize}
\textbf{	Метод перебора }- простейший из методов поиска значений действительно-значных функций по какому-либо из критериев сравнения (на максимум, на минимум, на определённую константу). Применительно к экстремальным задачам является примером прямого метода условной одномерной пассивной оптимизации.\\
	\vspace{\baselineskip}
	
	Основные преимущества метода:
	\begin{itemize}
		\item Достаточно прост в реализации.
		\item Показывает отличные результаты с определенной точностью. Не уступает симплекс-методу.
	\end{itemize}

Но есть и занчительный недостаток:
\begin{itemize}
	\item Данный метод не является удачным для решения объемных задач. Перебор больших матриц будет рассматривать слишком много комбинации, что приведет к значительному замедлению процесса решения задачи.
\end{itemize}
	
\end{document}