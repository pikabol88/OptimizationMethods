\documentclass[../body.tex]{subfiles}
\begin{document}
	\textbf{Теорема}\\
	Чтобы вектор $x[N]$ был решением исходной задачи в канонической форме, необходимо и достаточно, чтобы существовал положительный вектор $Y_{*}[M]$, являющийся решением двойственной задачи и удовлетворяющий следующим условиям:
	\begin{equation}\label{eq:first}
		Y_{*}[M_1]\ge 0
	\end{equation}
	\begin{equation}\label{eq:second}
		C^{T}[N_1]-Y^{T}_{*}[M]A[M,N_1]\ge 0
	\end{equation}
	\begin{equation}\label{eq:third}
		C^{T}[N_2]-Y^{T}_{*}[M]A[M,N_2] = 0
	\end{equation}
	\begin{equation}\label{eq:fourth}
		Y^{T}_{*}[M_1](A[M_1]X_{*}[N]-b[M_1] = 0
	\end{equation}
	\begin{equation}\label{eq:fifth}
		(C^{T}[N_1]-Y^{T}_{*}[M]A[M,N_1])*X_{*}[N_1] = 0)
	\end{equation}

	Проверим полученные результаты:\\
	
	Для нашей задачи:\\
    $$X^T =  \begin{pmatrix} 	0 & \frac{12}{119} & \frac{1}{119} & \frac{23}{119} & 1\frac{75}{119} & 0 & 0 & 1\frac{103}{119} \\  \end{pmatrix}$$
	\setcounter{MaxMatrixCols}{20}
	$$Y^T =  \begin{pmatrix}	1\frac{71}{119} & 0 & 0 & \frac{106}{119} & 1\frac{33}{119} & 1\frac{10}{119} & 0 & 0 & 2\frac{25}{119} & 0 & 0 & 0\\ \end{pmatrix}$$
	\vspace{\baselineskip}
	
	В исходных переменных:\\ 
	$$X^T_{*} =  \begin{pmatrix}	0 & \frac{12}{119} & \frac{1}{119} & \frac{23}{119} & \frac{194}{119} \\ \end{pmatrix}$$
	\setcounter{MaxMatrixCols}{20}
	$$Y^T_{*} =  \begin{pmatrix}		\frac{190}{119} & \frac{-106}{119} & \frac{-152}{119} & \frac{129}{119}  & 0\\ \end{pmatrix}$$
	\vspace{\baselineskip}
	$$ A[M,N] = 	\begin{pmatrix} 
		1 & 2 & 3 & 4 & 0 \\
		2 & 3 & 8 & 0 & 1 \\
		1 & 4 & 0 & 5 & 1 \\
		3 & 7 & 4 & 0 & 2 \\
		-2 & -3 & -5 & -6 & -1 \\
	\end{pmatrix}, 
	b[M]=\begin{pmatrix}
	1\\2\\3\\4\\
	-\frac{373}{119}
	\end{pmatrix}, 
	C^T[N]=
	\begin{pmatrix}
	4\\3\\2\\0\\0
	\end{pmatrix}\\$$ 
	\vspace{\baselineskip}
	
	По условию задачи:\\
	$$M_1=\left\lbrace 1,5\right\rbrace , M_2=\left\lbrace 2,3,4\right\rbrace $$
	$$N_1=\left\lbrace 1,2,3,4\right\rbrace , N_2=\left\lbrace 5\right\rbrace $$\\
	\vspace{\baselineskip}
	\textbf{Проверим выполнение условий (\ref{eq:first})-(\ref{eq:fifth}):}\\
	(\ref{eq:first})$
	\begin{pmatrix}
	\frac{190}{119} & 0
	\end{pmatrix}  \ge 0 $ 
	\checkmark \\
	\vspace{\baselineskip}
	
	(\ref{eq:second})
	$\begin{pmatrix}
		4\\ 3\\  2\\  0\\   
	\end{pmatrix} -
	\begin{pmatrix}	
		\frac{190}{119} & \frac{-106}{119} & \frac{-152}{119} & \frac{129}{119}  & 0\\
	\end{pmatrix} *  
		\begin{pmatrix} 
		1 & 2 & 3 & 4  \\
		2 & 3 & 8 & 0  \\
		1 & 4 & 0 & 5  \\
		3 & 7 & 4 & 0  \\
		-2 & -3 & -5 & -6  \\
		\end{pmatrix} =\\=
	\begin{pmatrix}
		4\\ 3\\  2\\  0\\  
	\end{pmatrix} -
	\begin{pmatrix}
		\frac{213}{119}\\[0.2cm] 3\\[0.2cm]  2\\[0.2cm]  0\\   
	\end{pmatrix} = 
	\begin{pmatrix}
	\frac{263}{119}\\[0.2cm] 0\\[0.2cm]  0\\[0.2cm]  0\\[0.2cm]   
	\end{pmatrix} \ge 0
	 $
	 \checkmark\\
	\vspace{\baselineskip}
	
	(\ref{eq:third}) 
	$\begin{pmatrix}
		0  
	\end{pmatrix} -
	\begin{pmatrix}	
		\frac{190}{119} & \frac{-106}{119} & \frac{-152}{119} & \frac{129}{119}  & 0\\
	\end{pmatrix} *  
	\begin{pmatrix} 
	0\\1\\1\\2\\-1\\
	\end{pmatrix} =(0)-(0)=0  $
	\checkmark\\
	\vspace{\baselineskip}
	
	(\ref{eq:fourth})
	$ 
	\begin{pmatrix}
		\frac{190}{119} & 0
	\end{pmatrix} * \left(
	\begin{pmatrix}
	1 & 2 & 3 & 4 & 0\\
	-2 & -3 & -5 & -6 & -1\\ 
	\end{pmatrix}*
	\begin{pmatrix}
		0 \\[0.2cm] \frac{12}{119} \\[0.2cm] \frac{1}{119} \\[0.2cm] \frac{23}{119} \\[0.2cm] \frac{194}{119} \\
	\end{pmatrix} - 
	\begin{pmatrix}
	\frac{-373}{119}
	\end{pmatrix}
	 \right)= \\=
	\begin{pmatrix}
		\frac{190}{119} &  0
	\end{pmatrix}*
	\left(
	\begin{pmatrix}
		1\\[0.2cm] \frac{-373}{119} 
	\end{pmatrix}
	-
	\begin{pmatrix}
			1\\[0.2cm] \frac{-373}{119} 
	\end{pmatrix}
	 \right) = 0
	 $
	 \checkmark\\
	\vspace{\baselineskip}
	
	(\ref{eq:fifth})
	$
	 \begin{pmatrix}
	\frac{263}{119}\\[0.2cm] 0\\[0.2cm]  0\\[0.2cm]  0\\[0.2cm]   
	\end{pmatrix} *
	 \begin{pmatrix}
	0 \\[0.2cm] \frac{12}{119} \\[0.2cm] \frac{1}{119} \\[0.2cm] \frac{23}{119} \\[0.2cm] \frac{194}{119} \\
	\end{pmatrix} =0 $
	 \checkmark\\
	 \vspace{\baselineskip}
	 
	 $\Longrightarrow $
	 Все условия выполняются и $X_*^T$ оптимальное решение при $\exists Y_*^T$  
\end{document}