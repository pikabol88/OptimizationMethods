\documentclass[../body.tex]{subfiles}
\begin{document}
\subsection{Алгоритм метода дихотомии}
\begin{enumerate}
    \item Вводим константу различимости $\alpha=\frac{b-a}{100}$
    \item На каждом шаге процесса поиска делим отрезок $[a,b]$ пополам, $x=\frac{a+b}{2}$ - координата середины отрезка $[a,b]$
    \item Вычисляем значение функции $F(x)$ в окрестности $\pm \alpha$ вычисленной точки $x$, т.е.
        \begin{equation}
            F_1=F(x-\alpha),\; F_2=F(x+\alpha)
        \end{equation}
    \item Сравниваем $F_1$ и $F_2$ и отбрасываем одну из половинок отрезка $[a,b]$
         \begin{itemize}
            \item Если $F_1<F_2$, то отбрасываем отрезок $[x,b]$, тогда $b=x$
            \item Иначе отбрасываем отрезок $[a,x]$, тогда $a=x$
        \end{itemize}
    \item Деление отрезка $[a,b]$ продолжается, пока его длина не станет меньше заданной точности $\varepsilon$, т.е. $|b-a| \le \varepsilon$
\end{enumerate}

\subsection{Алгоритм метода парабол}
\begin{enumerate}
    \item Определить начальные точки $x_1=a,x_2=\frac{a+b}{2},x_3=b$
    \item Вычислить значение функции цели $f_1, f_2, f_3$ в этих точках
    \item Вычислить коэффициенты $a_0=f_1,a_1=\frac{f_2-f_1}{x_2-x_1},a_2=\frac{1}{x_3-x_2}*(\frac{f_3-f_1}{x_3-x_1}-\frac{f_2-f_1}{x_2-x_1}$
    \item Вычислить новое значение точки минимума $x_*=0.5*(x_2+x_1-\frac{a_1}{a_2})$ и значение функции цели $f_*(x_*)$
        \begin{itemize}
            \item Если расстояние между новым значением точки минимума и полученным на прошлой итерации меньше заданной точности, получаем результат
            \item Если расстояние больше точности, то вычисляем новые точки $x1, x2, x3$ (обращений к функции цели нет, потому что используются $f1, f2, f3, f_*$) и возвращаемся к пункту 2
        \end{itemize}
\end{enumerate}
\end{document}