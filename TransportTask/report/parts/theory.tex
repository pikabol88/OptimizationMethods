\documentclass[../body.tex]{subfiles}
\begin{document}
\subsection{Применимость для метода потенциалов}
Для того чтобы транспортная задача была поставлена корректно, необходимо выполнение следующих условий:
\begin{itemize}
    \item $x_{ij}\geq 0$, где $i=\overline{1,n}, j=\overline{1,m}$
    \item $\sum_{i=1}^{n} a_i \geq \sum_{j=1}^{m} b_i$
\end{itemize}

Для того чтобы применять заданные методы к решению транспортной задачи, она должна быть приведена к закрытому виду.

Проверим эти условия:
\begin{enumerate}
    \item $x_{ij}\geq 0$, где $i=\overline{1,n}, j=\overline{1,m}$, задается в коде программы при определении данных переменных
    \item $\sum_{i=1}^{n} a_i=19+5+21+9=54$ и $\sum_{j=1}^{m} b_i=12+12+11+8+11=54$. Следовательно, задача задана в закрытом виде
\end{enumerate}

Таким образом, заданные методы применимы к данной задаче.

\subsection{Применимость для метода перебора крайних точек}
Для того чтобы применить данный метод к задаче, заданной в табличном виде, нужно преобразовать условие к СЛАУ и привести задачу к каноничному виду.
\begin{equation}
    \left\{
    \begin{array}{ll}
         x_1+x_2+x_3+x_4+x_5=19\\
         x_6+x_7+x_8+x_9+x_{10}=5\\
         x_{11}+x_{12}+x_{13}+x_{14}+x_{15}=21\\
         x_{16}+x_{17}+x_{18}+x_{19}+x_{20}=9\\
         x_1+x_6+x_{11}+x_{16}=12\\
         x_2+x_7+x_{12}+x_{17}=12\\
         x_3+x_8+x_{13}+x_{18}=11\\
         x_4+x_9+x_{14}+x_{19}=8\\
         x_i \geq 0, i=\overline{1,n}, n=20\\
    \end{array}
    \right.
\end{equation}

Функция цели примет вид $\sum_{i=1}^n{c_ix_i}\rightarrow min$, где $c_i$ - соответсвующие коэффициенты транспортной таблицы.
\end{document}