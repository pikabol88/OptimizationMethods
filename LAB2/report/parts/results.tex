\documentclass[../body.tex]{subfiles}
\begin{document}
\subsection{Результат нахождения плана методом потенциалов для исходной задачи}
Оптимальный опорный план имеет вид:
\begin{table}[h]
    \centering
    \begin{tabular}{|c|c|c|c|c|c||c||c|}
        \hline
        & $b_1$ & $b_2$ & $b_3$ & $b_4$ & $b_5$ & & $u_i$ \\\hline
        $a_1$ & 7 & 4 & - & 8 & - & 19 & 0\\\hline
        $a_2$ & 5 & - & - & - & - & 5 & -1\\\hline
        $a_3$ & - & 8 & 11 & - & 2 & 21 & 2\\\hline
        $a_4$ & - & - & - & - & 9 & 9 & -6\\\hline
        & 12 & 12 & 11 & 8 & 11 & &\\\hline
        \hline
        $v_j$ & 3 & 2 & 5 & 11 & 9 & &\\\hline
    \end{tabular}
    \caption{Оптимальный опорный план, найденный методом потенциалов}
    \label{tab:potentials}
\end{table}

Минимальные затраты при осуществлении данного опорного плана перевозок: $F(x)=3*7+2*4+11*8+2*5+4*8+7*11+11*2+3*9=285$

\subsection{Результат нахождения плана методом перебора крайних точек для исходной задачи}
В результате работы программы получено решение следующее решение прямой задачи:
\begin{equation}
    X_*^T=(7,12,0,0,0,5,0,0,0,0,0,0,11,0,10,0,0,0,8,1)
\end{equation}

Тогда функция цели принимает значение:$F(x)=3*7+2*12+2*5+7*11+11*10+5*8+3*1=285$

Как можно увидеть, данный оптимальный опорный план отличается от найденного методом потенциалов, но функция цели принимает то же значение. Такую ситуацию можно объяснить наличием нескольких допустимых оптимальных планов, и в случае метода перебора для составлении СЛАУ с линейно независимыми строками матрицы убиралось уравнение, состовляющее последний столбец транспортной таблицы. То есть при удалении другой строки, метод перебора даст другой допустимый оптимальный план.

К тому же стоит сказать, что одним из допустимых решений прямой задачи является оптимальный план:
\begin{equation}
    X_*^T=(7,4,0,8,0,5,0,0,0,0,0,8,11,0,2,0,0,0,0,9)
\end{equation}

В таком случае функция цели принимает значение: $F(x)=3*7+2*4+11*8+2*5+4*8+7*11+11*2+3*9=285$

То есть можно сделать вывод, что оба метода верно находят решение исходной задачи.

\subsection{Результат нахождения плана методом потенциалов для задачи с усложнением}
Оптимальный опорный план задачи с усложнением имеет вид:
\begin{table}[h]
    \centering
    \begin{tabular}{|c|c|c|c|c|c||c||c|}
        \hline
        & $b_1$ & $b_2$ & $b_3$ & $b_4$ & $b_5$ & & $u_i$ \\\hline
        $a_1$ & 13 & 4 & - & 2 & - & 19 & 0\\\hline
        $a_2$ & - & - & - & - & 5 & 5 & -1\\\hline
        $a_3$ & - & 10 & 11 & - & - & 21 & 2\\\hline
        $a_4$ & - & - & - & 2 & 7 & 9 & -6\\\hline
        $a_5$ & - & - & - & 6 & - & 9 & -2\\\hline
        & 13 & 14 & 11 & 10 & 12 & &\\\hline
        \hline
        $v_j$ & 3 & 2 & 5 & 11 & 9 & &\\\hline
    \end{tabular}
    \caption{Оптимальный опорный план задачи с усложнением}
    \label{tab:potentials}
\end{table}

Минимальные затраты при осуществлении данного опорного плана перевозок: $F(x)=3*13+2*4+11*2+8*5+4*10+7*11+5*2+3*7+9*6=311$

Добавим к итоговой стоимости перевозки для исходной задачи сумму штрафа за недопоставку: $G(x)=285+9*6=339$, то есть $F(x) < G(x)$, а значит, произошло перераспределение значений опорного вектора для уменьшения общей стоимости перевозки.
\end{document}