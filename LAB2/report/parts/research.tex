\documentclass[../body.tex]{subfiles}
\begin{document}
\subsection{Вывод цикла пересчета итерации метода потенциалов}
Рассмотрим вторую итерацию метода потенциалов для решения исходной задачи, для которой опорный план имеет следующий вид:
\begin{table}[h]
    \centering
    \begin{tabular}{|c|c|c|c|c|c||c||c|}
        \hline
        & $b_1$ & $b_2$ & $b_3$ & $b_4$ & $b_5$ & & $u_i$\\\hline
        $a_1$ & 12 & 7 & - & - & - & 19 & 0\\\hline
        $a_2$ & - & 5 & - & - & 0 & 5 & 2\\\hline
        $a_3$ & - & - & 11 & 8 & 2 & 21 & 5\\\hline
        $a_4$ & - & - & - & - & 9 & 9 & -3\\\hline
        & 12 & 12 & 11 & 8 & 11 & &\\\hline
        $v_j$ & 3 & 2 & 2 & 10 & 6 & &\\\hline
    \end{tabular}
    \caption{Опорный план на второй итерации}
\end{table}

Данный опорный план не является оптимальным, так как существуют свободные ячейки, для которых $v_j+u_i-c_{ij}>0$:
\begin{itemize}
    \item $(2,1):3+2-2=3$
    \item $(3,2):2+5-4=3$
    \item $(4,4):10-3-5=2$
\end{itemize}

Следовательно, $max{\{3,3,2\}}=3$ и выбираем ячейку $(2,1)$.
\begin{figure}[h]
    \center{\includegraphics[scale=0.5]{parts/cycle.jpg}}
\caption{Вывод цикла пересчета второй итерации}
\label{fig:image}
\end{figure}

Перемещение по таблице начинается с движения в правую сторону и обход продолжается по часовой стрелке. Из небазисной ячейки $(2,1)$ мы движемся до базисной ячейки $(2,2)$. Далее возможны движения вправо к небазисной ячейке $(2,3)$, вниз к небазисной ячейке $(3,2)$ и вверх к базисной ячейке $(1,2)$.

Рассмотрим, например, ячейку $(2,3)$. От нее мы движемся до базисной ячейки $(2,5)$, из которой движение вверх не приводит к результату, так как выше расположены только небазисные ячейки и невозможно сменить направление движения. Аналогичная ситуация происходит и при движении вниз от данной ячейки. При рассмотрении ячейки $(3,2)$ тоже не происходит смены направления движения. Значит, нам подходит ячейка $(1,2)$, из которой движение вправо не приносит результата, а движение влево к базисной ячейке $(1,1)$ вызывает смену направления на начальную ячейку $(2,1)$ и замыкание цикла.

\subsection{Решение траспортной задачи с усложнением}
Сформулируем задачу с усложнением. Пусть за недопоставку некоторого уровня груза начисляется соответсвующий штраф. 
\begin{table}[h]
    \centering
    \begin{tabular}{|c||c|c|c|}
        \hline
        Уровень & 1 & 5 & 10\\\hline
        Штраф & 3 & 9 & 20\\\hline
    \end{tabular}
    \caption{Штраф за недопоставку грузка}
    \label{tab:penalty}
\end{table}

Создадим ситуацию недопоставки, то есть увеличим потребности на несколько единиц. В таком случае сумма всех потребностей будет превышать сумму всех запасов, то есть задача имеет открытый вид.

\begin{table}[h]
    \centering
    \begin{tabular}{|c|c|c|c|c|c||c|}
        \hline
        & $b_1$ & $b_2$ & $b_3$ & $b_4$ & $b_5$ & \\\hline
        $a_1$ & 3 & 2 & 7 & 11 & 11 & 19\\\hline
        $a_2$ & 2 & 4 & 5 & 14 & 8 & 5\\\hline
        $a_3$ & 9 & 4 & 7 & 15 & 11 & 21\\\hline
        $a_4$ & 2 & 5 & 1 & 5 & 3 & 9\\\hline
        & 13 & 14 & 11 & 10 & 12 & \\\hline
    \end{tabular}
    \caption{Транспортная задача с усложнением}
    \label{tab:difficult_task}
\end{table}

В таком случае для использования метода потенциалов необходимо приведение задачи к закрытому виду, при этом в зависимости от значения разности потребностей и запасов вводится фиктивная величина, имеющая стоимость, соответсвующую уровню вычисленной разности. 
\end{document}