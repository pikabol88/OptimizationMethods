\documentclass[../body.tex]{subfiles}
\begin{document}
Поставлена задача линейного программирования, состоящая из пяти переменных, включающая три равенства и два неравенства разных знаков. На знаки для четырёх переменных поставлены ограничения:
\begin{equation}\label{eq:limitation}
\left\{
\begin{array}{ll} 
	x_1+2x_2+3x_3+4x_4 \ge 1\\
	2x_1+3x_2+8x_3+x_5=2\\
	x_1+4x_2+5x_4+x_5=3\\
	3x_1+7x_2+4x_3+2x_5 = 4\\
	2x_1+3x_2+5x_3+6x_4+x_5 \le 5\\
	x_1,x_2,x_3,x_4 \ge 0
\end{array}
\right.
\end{equation}

Функция цели:
\begin{equation}\label{eq:goal}
F(x) =  4x_1 + 3x_2 + 2x_3 \longrightarrow min 
\end{equation}
\begin{enumerate}
	\item Привести задачу к виду, необходимому для применения симплекс-метода.
	\item Построить к данной задаче двойственную и также привести к виду, необходимому для применения симплекс-метода.
	\item Автоматизировать привидение исходной задачи к каноническому виду.
	\item Решить обе задачи симплекс-методом с выбором начального приближения методом искусственного базиса.
	\item Решить обе задачи методом перебора крайних точек.
\end{enumerate}
	\textbf{Симплекс-метод} является классическим методом решения задач линейного программирования, который на практике зачастую бывает очень быстрым.
\end{document}

