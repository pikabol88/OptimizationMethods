\documentclass[../body.tex]{subfiles}
\begin{document}

 \subsection{Результат нахождения задачи двойственной к заданной}
	\begin{comment}
	Меняем знаки у ограничений с $\geq$, путём умножения на -1:
	\begin{equation}
			\left\{
		\begin{array}{ll} 
			x_1+2x_2+3x_3+4x_4 \ge 1\\
			2x_1+3x_2+8x_3+x_5=2\\
			x_1+4x_2+5x_4+x_5=3\\
			3x_1+7x_2+4x_3+2x_5 = 4\\
			2x_1+3x_2+5x_3+6x_4+x_5 \le 5\\
			x_1,x_2,x_3,x_4 \ge 0
		\end{array}
		\right.
		\Longrightarrow
		\left\{
		\begin{array}{ll} 
			-x_1-2x_2-3x_3-4x_4 \le -1\\
			2x_1+3x_2+8x_3+x_5=2\\
			x_1+4x_2+5x_4+x_5=3\\
			3x_1+7x_2+4x_3+2x_5 = 4\\
			2x_1+3x_2+5x_3+6x_4+x_5 \le 5\\
			x_1,x_2,x_3,x_4 \ge 0
		\end{array}
		\right.
	\end{equation}\\
	Для каждого ограничения с неравенством добавляем дополнительные переменные $x_6$ и $x_7$.
	Перепишем ограничения в каноническом виде:
	\begin{equation}
		\left\{
	\begin{array}{ll} 
		-x_1-2x_2-3x_3-4x_4\le -1\\
		2x_1+3x_2+8x_3+x_5=2\\
		x_1+4x_2+5x_4+x_5=3\\
		3x_1+7x_2+4x_3+2x_5 = 4\\
		2x_1+3x_2+5x_3+6x_4+x_5 \le 5\\
		x_1,x_2,x_3,x_4 \ge 0
	\end{array}
	\right.
		\Longrightarrow
		\left\{
			\begin{array}{ll} 
			-x_1-2x_2-3x_3-4x_4+x_{6} = -1\\
			2x_1+3x_2+8x_3+x_5=2\\
			x_1+4x_2+5x_4+x_5=3\\
			3x_1+7x_2+4x_3+2x_5 = 4\\
			2x_1+3x_2+5x_3+6x_4+x_5 + x_{7} = 5\\
			x_1,x_2,x_3,x_4 \ge 0
		\end{array}
		\right.
	\end{equation}\\
	\end{comment}
	
	Найдём двойствунную задачу для прямой задачи (\ref{eq:limitation}):	
	\begin{equation} \label{eq:dualtask}
		\left\{
		\begin{array}{ll} 
			x_1+2x_2+3x_3+4x_4 \ge 1\\
			2x_1+3x_2+8x_3+x_5=2\\
			x_1+4x_2+5x_4+x_5=3\\
			3x_1+7x_2+4x_3+2x_5 = 4\\
			2x_1+3x_2+5x_3+6x_4+x_5 \le 5\\
			x_1,x_2,x_3,x_4 \ge 0
		\end{array}\\
		\right.
		\Longrightarrow
		\left\{
		\begin{array}{ll} 
			x_1+2x_2+x_3+3x_4+2x_5\le 4\\
			2x_1+3x_2+4x_3+7x_4+3x_5 \le 3\\
			3x_1+8x_2+4x_4+5x_5 \le 2\\
			4x_1+5x_3+6x_5 \le 0\\
			x_2+x_3+2x_4+x_5 = 0\\
			x_1\ge 0, x_5 \le 0
		\end{array}
		\right.	
	\end{equation}\\
	Функция цели: $$F(x)=x_1+2x_2+3x_3+4x_4+5x_5\longrightarrow \max
	$$
	 \subsection{Результат приведения задач линейного программирования к каноническому виду}
	\begin{itemize}
		\item Приведём задачу (\ref{eq:limitation}) к каноническому виду:\\
	\begin{equation}
		\left\{
		\begin{array}{ll} 
			x_1+2x_2+3x_3+4x_4 \ge 1\\
			2x_1+3x_2+8x_3+x_5=2\\
			x_1+4x_2+5x_4+x_5=3\\
			3x_1+7x_2+4x_3+2x_5 = 4\\
			2x_1+3x_2+5x_3+6x_4+x_5 \le 5\\
			x_1,x_2,x_3,x_4 \ge 0
		\end{array}
		\right.
		\Longrightarrow
		\left\{
		\begin{array}{ll} 
			x_1+2x_2+3x_3+4x_4 -x_7 = -1\\
			2x_1+3x_2+8x_3+x_5-x_6=2\\
			x_1+4x_2+5x_4+x_5-x_6=3\\
			3x_1+7x_2+4x_3+2x_5-2x_6 = 4\\
			2x_1+3x_2+5x_3+6x_4+x_5-x_6 +x_8= 5\\
			x_1,x_2,x_3,x_4,x_5,x_6,x_7,x_8 \ge 0
		\end{array}
		\right.
	\end{equation}\\
	Функция цели:
	$$F(x) =  4x_1 + 3x_2 + 2x_3 \longrightarrow min $$
	
\item Приведём двойственную задачу (\ref{eq:dualtask}) к каноническому виду:
\begin{multline} 	
	\left\{
\begin{array}{ll} 
	x_1+2x_2+x_3+3x_4+2x_5\le 4\\
	2x_1+3x_2+4x_3+7x_4+3x_5 \le 3\\
	3x_1+8x_2+4x_3+5x_5 \le 2\\
	4x_1+5x_3+6x_5 \le 0\\
	x_2+x_3+2x_4+x_5 = 0\\
	x_1\ge 0, x_5 \le 0
\end{array}
	\right. 
	\Longrightarrow\\
	\left\{
	\begin{array}{ll} 
		x_1+2x_3-2x_4-x_5+3x_6-3x_7-2x_8+x_9= 4\\
		2x_1+3x_3-3x_4-4x_5+7x_6-7x_7-3x_8+x_{10}= 3\\
		3x_1+8x_3-8x_4+4x_6-4x_7-5x_8+x_{11}= 2\\
		4x_1-5x_5-6x_8+x_{12}= 0\\
		x_3-x_4-x_5+2x_6-2x_7-x_8= 0\\
		x_1,x_2,x_3,x_4,x_5,x_6,x_7,x_8,x_9,x_{10},x_{11},x_{12}\ge 0\\
	\end{array}
	\right.
\end{multline}

Функция цели: $$F(x)=x_1+2x_3-2x_4-3x_5+4x_6-4x_7-5x_8\longrightarrow \max
$$	
 \subsection{Результат решения прямой и двойственной задач линейного программирования}
\item Решение прямой задачи методом перебора крайних точек: 
$$x^{*}=(0, 0.10084403, 0.00084034, 0.19327731, 1.63025211, 0, 0, 1.86554621)$$
\item Решение двойственной задачи методом перебора крайних точек: 
$$x^{*}=(1.59664, 0, 0, 0.890756, 1.27731, 1.08403, 0,0,2.21008, 0,0,0)$$
\item Решение прямой задачи симплекс - методом:
$$x^{*}=(0, 0.10084403, 0.00084034, 0.19327731, 1.63025211, 0, 0, 1.86554621)$$
\item  Решение двойственной задачи симплекс - методом:
$$x^{*}=(1.59664, 0, 0, 0.890756, 1.27731, 1.08403, 0,0,2.21008, 0,0,0)$$

Можно заметить, что мы поличили одинаковый ответ, решая разными методами. Что указывает на корректность запрограммированного алгоритма и вычислений.

\end{itemize}
	


\end{document}