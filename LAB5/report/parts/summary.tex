\documentclass[../body.tex]{subfiles}
\begin{document}
Метод имеет ряд преимуществ. Во-первых, он гарантирует сходимость к оптимальной точке при выполнении основных условий применимости метода. К тому же решение вспомогательной задачи линейного программирования можт производиться любым наиболее подходящим под условия методом. Также учитываются при построении направления следующего шага только ограничения, близкие к активным, а на остальных ограничениях можно рассматривать задачу безусловной минимизации. Стоит сказать, что выбор величины шага позволяет оценить убывание значения функции в следующей точке по отношению к предыдущему шагу.

Недостатками данного метода могут являться достаточно жесткие условия применимости метода, налагаемые на функцию и область ее исследования, и сложность аналитического определения параметра дробления исходя из обстоятельств подбора шага и необходимости отодвигаться поближе к границе.

Зачастую в задачах на выбор начального приближения из рассматриваемого можества сложно подобрать такую точку, чтобы учесть все наборы ограничений. Для этого может использоваться метод всевозможных направлений Зойтендейка, позволяющий найти допустимую точку и выбрать ее в качестве начального приближения.
\end{document}