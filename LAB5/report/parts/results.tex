\documentclass[../body.tex]{subfiles}
\begin{document}
Для рассмотрения начальной стадии работы алгоритма взята точка $(-0.2,-0.4)$ и найдена допустимая точка для множества при $\eta_0=-2.29289322$.

\subsection{Решение задачи при внутренней оптимальной точке}
Решение задачи многомерной минимизации при ограничениях такого вида, что оптимальная точка $x^*$ находится внутри рассматриваемой области, с точностью $\varepsilon=10^{-5}$:
\begin{table}[H]
    \centering
    \begin{tabular}{|c|c|}
        \hline
        $x^*$ & (-0.42884464, -0.32163181)\\\hline
        $\phi_0^*$ & 3.10912635\\\hline
        Число итераций & 35\\\hline
    \end{tabular}
\end{table}

\begin{figure}[H]
    \centering
    \includegraphics[scale=0.7]{parts/answer1.jpg}
    \caption{Иллюстрация работы алгоритма}
\end{figure}

\subsection{Решение задачи при граничной оптимальной точке}
Решение задачи многомерной минимизации при ограничениях такого вида, что оптимальная точка $x^*$ находится на границе рассматриваемой области, с точностью $\varepsilon=10^{-3}$:
\begin{table}[H]
    \centering
    \begin{tabular}{|c|c|}
        \hline
        $x^*$ & (-0.42875498, -0.32138672)\\\hline
        $\phi_0^*$ & 3.10912645\\\hline
        Число итераций & 20\\\hline
    \end{tabular}
\end{table}

\begin{figure}[H]
    \centering
    \includegraphics[scale=0.8]{parts/answer2.jpg}
    \caption{Иллюстрация работы алгоритма}
\end{figure}
\end{document}